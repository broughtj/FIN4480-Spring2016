\documentclass[]{article}
\usepackage{lmodern}
\usepackage{amssymb,amsmath}
\usepackage{ifxetex,ifluatex}
\usepackage{fixltx2e} % provides \textsubscript
\ifnum 0\ifxetex 1\fi\ifluatex 1\fi=0 % if pdftex
  \usepackage[T1]{fontenc}
  \usepackage[utf8]{inputenc}
\else % if luatex or xelatex
  \ifxetex
    \usepackage{mathspec}
    \usepackage{xltxtra,xunicode}
  \else
    \usepackage{fontspec}
  \fi
  \defaultfontfeatures{Mapping=tex-text,Scale=MatchLowercase}
  \newcommand{\euro}{€}
\fi
% use upquote if available, for straight quotes in verbatim environments
\IfFileExists{upquote.sty}{\usepackage{upquote}}{}
% use microtype if available
\IfFileExists{microtype.sty}{%
\usepackage{microtype}
\UseMicrotypeSet[protrusion]{basicmath} % disable protrusion for tt fonts
}{}
\usepackage{longtable,booktabs}
\ifxetex
  \usepackage[setpagesize=false, % page size defined by xetex
              unicode=false, % unicode breaks when used with xetex
              xetex]{hyperref}
\else
  \usepackage[unicode=true]{hyperref}
\fi
\hypersetup{breaklinks=true,
            bookmarks=true,
            pdfauthor={},
            pdftitle={},
            colorlinks=true,
            citecolor=blue,
            urlcolor=blue,
            linkcolor=magenta,
            pdfborder={0 0 0}}
\urlstyle{same}  % don't use monospace font for urls
\setlength{\parindent}{0pt}
\setlength{\parskip}{6pt plus 2pt minus 1pt}
\setlength{\emergencystretch}{3em}  % prevent overfull lines
\setcounter{secnumdepth}{0}

\date{}

\begin{document}

\section{Finance 4480 - Derivatives
Markets}\label{finance-4480---derivatives-markets}

\begin{itemize}
 \itemsep1pt\parskip0pt\parsep0pt
 \item \textbf{Instructor:} Tyler J. Brough
 \item \textbf{Email:} \href{mailto:tyler.brough@aggiemail.usu.edu}{\nolinkurl{tyler.brough@aggiemail.usu.edu}}
 \item \textbf{Office:} BUS 605
 \item \textbf{Course Dates:} January 12 - April 28, 2016
 \item \textbf{Course Room:} \href{https://www.usu.edu/map/index.cfm?id=32}{ENGR 304}
 \item \textbf{Course Time:} TR 7:30 - 8:45 AM
 \item \textbf{Course Website:} \href{http://broughtj.github.io/teaching/FIN4480/}{http://broughtj.github.io/teaching/FIN4480/}  
 \item \textbf{Office hours:} TBD
\end{itemize}

\subsection{Course Description}\label{course-description}

This course is an introduction to derivative securities, the markets in
which they trade, theoretical models for obtaining rational prices for
them, and their applications in hedging and speculation. Derivatives are
an advanced finance topic. At one time, a course in derivatives was
considered a specialized elective, but is now considered mandatory in
any finance major. Our topic is by nature an analytical one that will
require a good deal of mathematics and quantitative reasoning. However,
we will seek a balance between the theoretical presentation of the
material and a practical implementation. The skill set necessary to
effectively implement modern financial derivative pricing models is a
crucial one to have for those seeking a job in the finance industry.

\subsection{Course Objectives}\label{course-objectives}

\begin{itemize}
\itemsep1pt\parskip0pt\parsep0pt
\item
  Learn fundamental principles and theories.

  \begin{itemize}
  \itemsep1pt\parskip0pt\parsep0pt
  \item
    Learn what derivatives are and their basic vocabulary.
  \item
    Learn how derivatives markets operate and the institutional features
    of the markets in which they trade.
  \item
    Learn the principles of modern financial economic theory explaining
    how derivative securities are priced.
  \end{itemize}
\item
  Learn to apply course materials

  \begin{itemize}
  \itemsep1pt\parskip0pt\parsep0pt
  \item
    Learn a framework for corporations, individuals, and other
    organizations to determine how and when to use derivative securities
    for hedging and speculation.
  \item
    Learn to implement derivative pricing models for applications in
    hedging and speculation.
  \end{itemize}
\item
  Developing specific skills, competencies and points of view needed by
  professionals in the finance industry and in academic finance.

  \begin{itemize}
  \itemsep1pt\parskip0pt\parsep0pt
  \item
    Learn the basics of \href{https://www.python.org/}{Python}
    programming for purposes of financial modeling.
  \end{itemize}
\end{itemize}

\subsubsection{Python Programming}\label{python-programming}

One unique objective of this course is to learn the basics of Python
programming for financial modeling. I have found that programming
financial models is an extremely helpful pedagogical tool for learning
financial theory. In addition, I have found that my past students have
benefited very much from learning some basic programming skills.

The version of Python that we will be using is the following:

\begin{itemize}
 \item The \href{https://store.continuum.io/cshop/anaconda/}{Anaconda}
  edition of Python by Continuum Analytics. This will have everything
  that we will need pre-installed. Please install the version that comes
  with Python 3. It is completely free.
\end{itemize}

Python is a very good choice for a programming language for this course.
As a programming language, Python is fairly easy to learn and can be put
to productive use very quickly. Python is quickly becoming the de facto
scripting language for scientific computing, and is widely used in the
finance industry. Programming will be used as a tool to reinforce
learning about derivatives and how they are priced, it will not be the
main focus of the course. The amount of programming that you will be
required to learn is quite modest. Nevertheless, I cannot emphasize
enough how important these skills are for a student of finance to be
competitive in the labor market.

\subsubsection{Linux/UNIX}\label{linuxunix}

I also \textbf{highly} recommend that you learn a little about the UNIX
command line as well. I will be covering some basic material about UNIX
in the course, and will use if heavily myself. While I encourage this,
your grade will not depend on this material at all.
But I think you will find it a very helpful addition to your skill set.

\subsection{Textbook}\label{textbook}

The \emph{required} textbook for the course is
\emph{\href{http://goo.gl/G5O2fx}{Options, Futures and Other Derivatives 9th Edition}}
by John C. Hull.

In addition, I will follow the book
\emph{\href{http://goo.gl/iipcHA}{Python Programming for the Absolute Beginner 3rd Edition}}, in my
presentation of the Python programming language. This book is
\textbf{not} required, but it is \textbf{strongly suggested} for those
who have never programmed before. 

While the first book doesn't have a very serious subject (simple
computer games) the programming taught is \emph{serious} programming and
it is taught in a way that takes away some of the intimidation
associated with other beginning programming books. I highly recommend
it!

\subsection{Grading}\label{grading}

The grade that you will earn will be determined from your ranking in the
class based on the weighted total points accumulated on class
preparation and participation, a class project, as well as on exams.
There is no predetermined percentage of the class that will earn an A or
that will fail. If you all do excellent work you will all earn
exceptional grades. The weights given to each part of the class are as
follows:

\subsubsection{Class Preparation \& Participation
(20\%)}\label{class-preparation-participation-20}

This portion of your grade will be determined by your preparation for
class lectures, your participation in class discussions and the
completion of homework assignments. Homework assignments will be given
approximately every Thursday and will be due a week later. I will drop
your lowest assignment and replace it with the average of the others if
you hand all assignments in with full effort. Participation in class is
crucial!

\subsubsection{Class Projects (40\%)}\label{class-projects-40}

This portion of your grade will be determined by a class project. The
project will consist of a computational modeling exercise. You can work
in teams of two. Potential topics will be discussed in class, but here
are some suggestions:

\begin{itemize}
\itemsep1pt\parskip0pt\parsep0pt
\item
  Monte Carlo pricing of an exotic option such as Asian option
\item
  Monte Carlo estimation of hedging strategies
\item
  Monte Carlo pricing of a swaption
\item
  Monte Carlo pricing of an option subject to stochastic volatility
\item
  Monte Carlo pricing of an option subject to price jumps
\item
  Monte Carlo evaluation of the Stutzer pricing methodology
\end{itemize}

\subsubsection{Exams (40\%)}\label{exams-40}

This portion of your grade will be determined by your performance on the
midterm and final exams. I will calculate your grade two ways:

\begin{itemize}
\itemsep1pt\parskip0pt\parsep0pt
\item
  The total of your two exam scores
\item
  Drop the midterm exam and place all of the weight on the final
\end{itemize}

Your grade will be determined by which ever method gives you the highest
total points.

\subsection{Format and Attendance}\label{format}

This course is designed to be as interactive and hands-on during our face-to-face class time as possible. As much as possible
I will seek to present the dense textbook material in online video lectures. You will watch these on your own time out of class.
We will use class time to dig deeper into the material, solve problems, and go over homework problems. This puts a strong burden
on each student to come to course prepared to participate in class. 

To make this work in practice class attendance will be mandatory. To enforce this I will collect a list of questions from each of you
at the beginning of class. Often I will use those questions to decide upon how we will spend class time. 

\newpage
\subsection{Topics (Subject to Change)}\label{topics-subject-to-change}

Please see the course website and the course Google Drive folder for a spreadsheet with
and updated schedule.


Important Dates:

\begin{itemize}
\itemsep1pt\parskip0pt\parsep0pt
\item
  \textbf{Feb 16} - Monday Schedule
\item
  \textbf{Mar 7 - 11} - Spring Break
\item
  \textbf{Apr 28} - Last Day of Classes
\item
  \textbf{Apr 3} - Final Exam
\end{itemize}

\end{document}
